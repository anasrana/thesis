In the model outlined in Chapter \ref{cha:mast} and applied to two biological systems in Chapters \ref{cha:oncog-transf} and \ref{cha:stem-cells} the initial cell population is assumed to be homogeneous. In the two applications discussed before this assumption is warranted due to experimental design. In the case of the oncogenic transformation the experiment is started from a cell line ensuring homogeneity. In the case of stem cell reprogramming the technique outlined by \cite{Hanna:2009ix} tries to ensure initial homogeneity  by using a secondary MEF cells.

Recently it has been shown that even seemingly homogeneous cell populations have an
inherent mixture be it at an epigenetic level \citep{Heng:2009em,Swanton:2012bm}. In this Chapter we setup a model that
answers the question: What effect does the initial cell population have on cell fate.



\begin{figure}[h]
  \centering
  
  \caption{Schematic of heterogeneous cell population transforming under stimulus.}
  \label{fig:cell-cycle-model}
\end{figure}

An example of such a biological system (schematic Figure \ref{fig:cell-cycle-model}) is
one with an initial heterogeneous cell population made up of two types of cells, with an
indistinguishable phenotype. At time $t=0$ the cells receive a stimulus leading to a
transformation such that at $t=T$ it is possible to distinguish cells in their final cell
fate. Now it is possible to count the fraction of cells that reach each of those final
cell fates. The interesting case here is when the strength of the stimulus has an affect on the
fraction of cells in each cell fate. We are interested in genes whose expression varies

% The question we try to answer in this chapter is, "what if the initial population is heterogeneous?". 

\section{Formalization}
\label{sec:formalization}

To formalize the model we start with 

%%% Local Variables:
%%% TeX-master: "warwickthesis"
%%% End: