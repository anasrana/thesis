% warwickthesis.tex modified by M Hadley from utthesis.doc  Sept 96
% Significant changes were made in 2009, first to work seemlessly with pdflatex
% and secondly to use the setspace package to control linespacing -
% removing some incompatibilities that existed before.
% any comments or problems - contact me  <m.j.hadley@warwick.ac.uk>
%%%%%%%%%%%%%%%%%%%%%%%%%%%%%%%%%%%%%%%%%%%%%%%%%%%%%%%%%%%%%%%%%%%%%%%%%%%%%
%%%
%%% File: utthesis.doc, version 2.0, January 1995
%%% =============================================
%%% Copyright (c) 1995 by Dinesh Das.  All rights reserved.
%%% This file is free and can be modified or distributed as long as
%%% you meet the following conditions:
%%%
%%% (1) This copyright notice is kept intact on all modified copies.
%%% (2) If you modify this file, you MUST NOT use the original file name.
%%%
%%% This file contains a template that can be used with the package
%%% utthesis.sty and LaTeX2e to produce a thesis that meets the requirements
%%% of the Graduate School of The University of Texas at Austin.
%%%
%%% All of the commands defined by utthesis.sty have default values (see
%%% the file
%           warwickthesis.sty
%%%                        for these values).  Thus, theoretically, you
%%% don't need to define values for any of them; you can run this file
%%% through LaTeX2e and produce an acceptable thesis, without any text.
%%% However, you probably want to set at least some of the macros (like
%%% \thesisauthor).  In that case, replace "..." with appropriate values,
%%% and uncomment the line (by removing the leading %'s).
%%%
%%%%%%%%%%%%%%%%%%%%%%%%%%%%%%%%%%%%%%%%%%%%%%%%%%%%%%%%%%%%%%%%%%%%%%%%%%%%%
% all comments starting with %! have been added by M Hadley as
% part of the conversion for the university of warwick
%
%
%\documentclass[11pt,a4paper,twoside]{report}      %% LaTeX2e document.
%%* Removed twoside option
\documentclass[11pt,a4paper]{report}      %% LaTeX2e document.
\usepackage{warwickthesis,setspace,graphicx,subfigure}     %!  setspace is used to control linepacing
\usepackage[square]{natbib}                    %! needed for Harvard style of references.
                                                %! for more notes see the bibliography section below
\usepackage{xcolor}
\usepackage{algorithm}
\usepackage{algorithmicx}
\usepackage{algpseudocode}
\usepackage{amsmath}
\usepackage{enumerate}
\usepackage{algorithm,algorithmicx,algpseudocode}
\usepackage[toc,page]{appendix}
\usepackage[T1]{fontenc}
%% Choose one of the following (if not choosing the default,
%% viz., Computer Modern, font family):
% \usepackage{lmodern}
% \usepackage{mathpazo}
% \usepackage{kpfonts}
% \usepackage{mathptmx}
% \usepackage{times,mtpro2}
% \usepackage{txfonts}
% \usepackage{newtxtext,newtxmath}
\usepackage{libertine}
% \usepackage[libertine]{newtxmath}
\usepackage[hang,bf]{caption}
\usepackage{hyperref}
% \mastersthesis                     %% Uncomment one of these; if you don't
\phdthesis                         %% use either, the default is \phdthesis.

%\thesisdraft                       %% Uncomment this if you want a draft
                                     %% version; this will print a timestamp
                                     %% on each page of your thesis.

 % \leftchapter                       %% Uncomment one of these if you want
 % \centerchapter                     %% left-justified, centered or
 \rightchapter                      %% right-justified chapter headings.
                                     %% Chapter headings includes the
                                     %% Contents, Acknowledgments, Lists
                                     %% of Tables and Figures and the Vita.
                                     %% The default is \centerchapter.

%\renewcommand{\familydefault}{cmss}  %! removed April 2009 because the default times font reads more easily
                                     %! for larger blocks of text.%!
                                     %! Added March 2003.
                                     %! This alternative is to use a sans serif font as in
                                     %!  the Warwick Corporate style.
                                     %! The default is Times, which is still acceptable.


\onehalfspacing                      %! This is the default and gives an acceptable "double spaced" thesis
                                     %! It is the minimum spacing accepted by the graduate school, and there is no reason to increase the spacing.
% \singlespacing                     %! Uncomment if you want single-spacing, 
% \doublespacing                     %! uncomment if you want real double-spacing for some perverse reason. 

%\setlength{\textheight}{9.0in}      %! Uncomment this for a slightly
                                     %! longer page. The default is now 8.5in (from Feb 2010)
                                     %! regulations require page numbers to be at least 1.5cm into the page.
                                     %! You can even try a longer page to save paper.

%! Double sided printing is no longer allowed (March 2008), it caused too many problems at binding,
                              %\setlength{\evensidemargin}{0.15in}  %! Uncomment this line for double sided printing
                                      %! Double-sided printing has recently been
                                      %! allowed by the Graduate School (March 2003)
                                      %! The default is {0.7in} for single sided.
%! Double sided printing is no longer allowed (March 2008), it caused too many problems at binding,
\DeclareMathOperator{\arcsinh}{arcsinh}
\renewcommand{\thesisdepartmentname}{Physics and Complexity Science}    %! The name of
                                                  %   the department

%! \renewcommand{\thesissubmission}{Submitted to the University of Warwick\\
%!              in partial fulfilment of the requirements\\
%!                   for admission to the degree of\\}
%!
%!!!!!!!! default is:
%!
\renewcommand{\thesissubmission}{Submitted to the University of Warwick\\
                        for the degree of}
%!
%! In the title page this wording will be preceeded by:  thesis\\
%!                 and ended by:  Doctor of Philosophy   (or the
%!                                               selected alternative names
%! use \\ where you want a new line

\renewcommand{\thesisauthor}{Anas A. Rana}    %% Your official name.

\renewcommand{\thesismonth}{...}     %% Your month of graduation.

\renewcommand{\thesisyear}{....}      %% Your year of graduation.

\renewcommand{\thesistitle}{Stochastic models for cell populations undergoing transitions}     %% The title of your thesis; use
                                     %% mixed-case.

%! \renewcommand{\thesistitletypesize}{\LARGE}   %! Put this in if you
                                  %!   want a Large title the default is \large

\renewcommand{\thesisauthorpreviousdegrees}{....}
                                     %% Your previous degrees, abbreviated;
                                     %% separate multiple degrees by commas.

\renewcommand{\thesissupervisor}{Sach Mukherjee, Matthew Turner}
                                     %% Your thesis supervisor; use mixed-case
                                     %% and don't use any titles or degrees.

\renewcommand{\thesisauthoraddress}{....}
                                     %% Your permanent address; use "\\" for
                                     %% linebreaks.
%! For the library declaration page only
%! \renewcommand{\thesiscopyrightagree}{agree}
                        %! agreement to allow single photocopies
%! \renewcommand{\thesiscopyrightagree}{do not agree}
                        %! refusal  to allow single photocopies
%! \renewcommand{\thesiscopyrightagree}{agree/do not agree}
                        %! undecided !!
                        %! default is agree


%%%%%%%%%%%%%%%%%%%%%%%%%%%%%%%%%%%%%%%%%%%%%%%%%%%%%%%%%%%%%%%%%%%%%%%%%%%%%
%%%
%%% The following commands are all optional, but useful if your requirements
%%% are different from the default values in utthesis.sty.  To use them,
%%% simply uncomment (remove the leading %) the line(s).

% \renewcommand{\thesisdegree}{...}  %% Uncomment this only if your thesis
                                     %% degree is NOT "DOCTOR OF PHILOSOPHY"
                                     %% for \phdthesis or "MASTER OF ARTS"
                                     %% for \mastersthesis.  Provide the
                                     %% correct FULL OFFICIAL name of
                                     %% the degree.

% \renewcommand{\thesisdegreeabbreviation}{...}
                                     %% Use this if you also use the above
                                     %% command; provide the OFFICIAL
                                     %% abbreviation of your thesis degree.

%\renewcommand{\thesistype}{Thesis}    %% Use this ONLY if your thesis type
                                     %! is NOT "Thesis"
                                     %% Provide the OFFICIAL type of the
                                     %% thesis; use mixed-case.

% \renewcommand{\thesistypist}{...}  %% Use this to specify the name of
                                     %% the thesis typist if it is anything
                                     %% other than "the author".

%%%
%%%%%%%%%%%%%%%%%%%%%%%%%%%%%%%%%%%%%%%%%%%%%%%%%%%%%%%%%%%%%%%%%%%%%%%%%%%%%


%\input header.tex          %! Input declarations, new
                              %theorems etc.


\begin{document}
\setlength{\parindent}{0cm}

%%* Made default
\thesiscopyrightpage                 %! Generate the copyright page.

%%* Uncomment a ttitle page.
 \thesistitlepage                     %% Generate the title page.
%\thesistitlecolourpage           %! Generates a COLOUR title page.

%%* Start roman page numbering here for contents, etc
\pagenumbering{roman} %! Begins roman numerals start from page i.

\tableofcontents                     %% Generate table of contents.
% \listoftables                      %% Uncomment this to generate list
                                     %% of tables.
% \listoffigures                     %% Uncomment this to generate list
                                     %% of figures.

\begin{thesisacknowledgments}        %% Use this to write your
%  \input ack.tex                    %% acknowledgments; it can be anything
                                     %% allowed in LaTeX2e par-mode.

                                     %! This following is not needed, but you may like to add it.
%This \lowercase\expandafter{\thesistype} was typeset with
%\LaTeXe\footnote{\LaTeXe{} is an extension of \LaTeX. \LaTeX{} is
%a collection of macros for \TeX. \TeX{} is a trademark of the
%American Mathematical Society. The style package {\em warwickthesis} was
%used.} by \thesistypist.

\end{thesisacknowledgments}

\begin{thesisdeclaration}        %! Use this to declare the extent of
                 %! the original work,
                 %! collaboration, other published
                                 %! material etc.it can be anything
                                 %% allowed in LaTeX2e par-mode.
Replace this text with a declaration of the extent of the original work,
collaboration, other published material etc. You can use any \LaTeX\
constructs.

\end{thesisdeclaration}


\begin{thesisabstract}               %% Use this to write your thesis
                                     %% abstract; it can be anything
                                     %% allowed in LaTeX2e par-mode.
%!  \begin{singlespace}       %! uncomment this if you need single spacing
%   \input abstract.tex       %!           don't forget the end spacing!
                                     %! It must fit on one page.
                                     %! single spacing and smaller
                                     %! font size
                                     %!  is allowed here.
%!   \end{singlespace}
\end{thesisabstract}

%\begin{thesisabbreviations}       %! Use this to give a list of
                                   %! abbreviateons
                                   %! It can be anything
%\end{thesisabbreviations}         %! allowed in LaTeX2e par-mode.
                                   %!The following may be useful':
                     %!\begin{itemize}
                     %!     \item[symbol]descriptive text..
                     %!\end{itemize}

%\end{thesisabbreviations}
%!!!!!!!!!!!!!!!                     %% Begin your thesis text here; follow
                                     %% the report style and group your text
                                     %% in chapters, sections, etc. eg:
%%* don't need this with one-sided printing
%\newpage{\pagestyle{empty}\cleardoublepage} %! ensure that Chapter 1 starts on an odd
                                           %! page when using double sided printing.
%%* Start arabic numbering of main text here
\pagenumbering{arabic} %! Begins arabic numerals start from page 1.

\chapter{Introduction}
\input introduction.tex


                            %% More chapters.

\chapter{Background}
\input background-math.tex

\input background-bio.tex

\chapter{State transitions using aggregated Markov models}
\label{cha:stamm}
\input stamm.tex

\chapter{Oncogenic Transformation}
\label{cha:oncog-transf}
\input mcf10a.tex

\chapter{Stem cells}
\label{cha:stem-cells}
\input reprogramming.tex

\chapter{Cell cycle}
\label{cha:cell-cycle}

\input cell-cycle.tex

\begin{appendices}
  \input appendix-stamm.tex
  \input appendix-oncogenic.tex

\end{appendices}

%!
%! There are a few variations of reference
% \begin{verbatim}\citet[chap. 2]{ballentine82}|
% \end{verbatim}
% for a textual one, as \citet[chap. 2]{ballentine82}.\\
%  \\
% \begin{verbatim}\citep{abraham_etal}
%  \end{verbatim}
%  for a parenthetical citation \citep{abraham_etal},\\

%  \begin{verbatim}\citep*{MTW}
%  \end{verbatim}
%  for a full list of authors use a * parenthetical citation \citep*{MTW},\\
%  \\
%!!!!!!!!!!!!!!!

%  \appendix                            %% this will do the appendices
%  \chapter{Proof of Fred's theorem}
%  \input{app1.tex}
%  \chapter{listing of Fred's program}
%  \input{app2.tex}

\bibliographystyle{plainnat}

\bibliography{fullbib}            %% Start your bibliography here;
                                 %! with sample.bib as your bibliography file. You can
                               %% also use:
                %! \begin{thebibliography}
                %!    \bibitem{etc....
                %! \end{thebibliography}
                               %% to generate your bibliography.

%\begin{thesisauthorvita}             %% Write your vita here; it can be
%                                     %% anything in LaTeX2e par-mode.
%\end{thesisauthorvita}               %%

\end{document}                       %% Done.
